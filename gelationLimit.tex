%This is the article LaTeX template for RSC journals
%Copyright The Royal Society of Chemistry 2010
% \usepackage{times}
% feel free not to use mathptmx if it causes difficulties
%eps figures can be used instead


\documentclass[8.5pt,twoside,twocolumn]{article}
%%%%%%%%%%%%%%%%%%%%%%%%%%%%%%%%%%%%%%%%%%%%%%%%%%%%%%%%%%%%%%%%%%%%%%%%%%%%%%%%%%%%%%%%%%%%%%%%%%%%%%%%%%%%%%%%%%%%%%%%%%%%%%%%%%%%%%%%%%%%%%%%%%%%%%%%%%%%%%%%%%%%%%%%%%%%%%%%%%%%%%%%%%%%%%%%%%%%%%%%%%%%%%%%%%%%%%%%%%%%%%%%%%%%%%%%%%%%%%%%%%%%%%%%%%%%
\usepackage{amssymb}
\usepackage{eurosym}
\usepackage[super,sort&compress,comma]{natbib}
\usepackage{mhchem}
\usepackage{times,mathptmx}
\usepackage{units}
\usepackage{sectsty}
\usepackage{balance}
\usepackage{graphicx}
\usepackage{epstopdf}
\usepackage{lastpage}
\usepackage[format=plain,justification=raggedright,singlelinecheck=false,font=small,labelfont=bf,labelsep=space]{caption}
\usepackage{fancyhdr}

%TCIDATA{OutputFilter=LATEX.DLL}
%TCIDATA{Version=5.50.0.2960}
%TCIDATA{<META NAME="SaveForMode" CONTENT="1">}
%TCIDATA{BibliographyScheme=BibTeX}
%TCIDATA{LastRevised=Sunday, December 22, 2013 00:50:11}
%TCIDATA{<META NAME="GraphicsSave" CONTENT="32">}

\oddsidemargin -1.2cm
\evensidemargin -1.2cm
\textwidth 18cm
\headheight 1.0in
\topmargin -3.5cm
\textheight 22cm


\begin{document}


\thispagestyle{plain} 
\fancypagestyle{plain}{
\renewcommand{\headrulewidth}{1pt}} \renewcommand{\thefootnote}{%
\fnsymbol{footnote}} 
\renewcommand\footnoterule{\vspace*{1pt}\hrule width
3.4in height 0.4pt \vspace*{5pt}} \setcounter{secnumdepth}{5}

\makeatletter

\renewcommand\@biblabel[1]{#1} \renewcommand\@makefntext[1]{\noindent%
\makebox[0pt][r]{\@thefnmark\,}#1} \makeatother
\renewcommand{\figurename}{\small{Fig.}~} \sectionfont{\large} %
\subsectionfont{\normalsize}

%\fancyfoot{} \fancyfoot[LO,RE]{\vspace{-7pt}%
%\includegraphics[height=9pt]{headers/LF}} \fancyfoot[CO]{\vspace{-7.2pt}%
%\hspace{12.2cm}\includegraphics{headers/RF}} \fancyfoot[CE]{\vspace{-7.5pt}%
%\hspace{-13.5cm}\includegraphics{headers/RF}} \fancyfoot[RO]{\footnotesize{%
%\sffamily{1--\pageref{LastPage} ~\textbar  \hspace{2pt}\thepage}}} 
%\fancyfoot[LE]{\footnotesize{\sffamily{\thepage~\textbar\hspace{3.45cm}
%1--\pageref{LastPage}}}} \fancyhead{} \renewcommand{\headrulewidth}{1pt} %
\renewcommand{\footrulewidth}{1pt} \setlength{\arrayrulewidth}{1pt} %
\setlength{\columnsep}{6.5mm} \setlength\bibsep{1pt}

\twocolumn[
  \begin{@twocolumnfalse}
\noindent\LARGE{\textbf{Osmotic pressure of endogenous mucus reveals robustness of human airway defense}}
\vspace{0.6cm}

\noindent\large{\textbf{Li-Heng Cai,\textit{$^{a,b,\dag}$} Brian Button,\textit{$^{c}$} Richard C. Boucher,\textit{$^{c}$} and
Michael Rubinstein$^{\ast}$\textit{$^{a,b}$}}}


\vspace{0.6cm}

\noindent \normalsize{\textbf{Abstract:} It was recently demonstrated that osmotic pressure of 
mucus is an important measure for the status of an innate human airway defense system -- mucus 
clearance, which is impaired at high mucus concentration for disease. We here 
report the results of systematic, quantitative measurements of endogenous mucus osmotic pressure. We find that the osmotic pressure 
of endogenous mucus is almost independent of pore size of the semipermeable membrane that separates mucus and reference
buffer solution, suggesting that there are few free molecules contained  in endogneous mucus. Moreoever, it increases 
as an unexpectedly slow linear rate with concentration up to a value comparable to that of mucus for 
chronic obstructive pulmonary disease (COPD). This slow increase of mucus osmotic pressure indicates 
that the mucus clearance can function within a wide range of mucus concentration, except for patients with lung diseases accompanied  with extremely high mucus concentration, such as cystic fibrosis (CF). 
Moreover, the consistency in osmotic pressure of endogenous mucus and clinical sputum suggests osmotic pressure of
mucus/sputum could be an important medical indication for the health status of lung.}
\vspace{0.5cm}
\end{@twocolumnfalse}
]

%Footnotes
%\footnotetext{%
%\dag ~Electronic Supplementary Information (ESI) available: [details of any
%supplementary information available should be included here]. See DOI:
%10.1039/b000000x/}

%Please use \dag to cite the ESI in the main text of the article.
%If you article does not have ESI please remove the the \dag symbol from the title and the above footnotetext.

\footnotetext{\textit{$^{a}$Department of Chemistry, $^{b}$Curriculum in
Applied Sciences and Engineering, $^{c}$Cystic Fibrosis Research Center,
School of Medicine, University of North Carolina, Chapel Hill, North
Carolina 27599, United States. $^{\dag}$Current address: School of Applied
Sciences and Engineering, Harvard University, Cambridge, Massachusetts
02138, United States. $^{\ast}$Correspondence: mr@unc.edu}}

%additional addresses can be cited as above using the lower-case letters, c, d, e... If all authors are from the same address, no letter is required

%\footnotetext{%
%\ddag ~Additional footnotes to the title and authors can be included \emph{%
%e.g.}\ `Present address:' or `These authors contributed equally to this
%work' as above using the symbols: \ddag , \textsection, and \P . Please
%place the appropriate symbol next to the author's name and include a \texttt{%
%\TEXTsymbol{\backslash}footnotetext} entry in the the correct place in the
%list.}

Mucus is a critical component of mucociliary clearance \cite{Wanner1996},
one of the major innate defense systems of human airways. The mucus protects
the pulmonary surfaces from inhaled toxicants and infectious agents by
trapping and transporting them out of lung by cilia-generated forces. The
muco-ciliary clearance works efficiently for healthy people but fails for
patients with lung diseases that are associated with concentrated mucus \cite%
{Matsui2006}, such cystic fibrosis (CF) and chronic obstructive pulmonary
disease (COPD) \cite{Boucher2007}. The failure of mucus clearance in disease
could be physically explained by our recent Gel-on-Brush model \cite%
{Button2012}that demonstrates that the concentrated mucus gel generates a
high osmotic pressure, dehydrates the brushlike periciliary layer (PCL) in
which the cilia are beating, and thus collapses the PCL, leading to impaired
mucus clearance, as illustrated in Figure \ref{fig:fig1_AirwayDefense}. In
spite of the importance of osmotic properties of mucus, to our knowledge
there is no systematic quantitative measurement of the osmotic pressure of
endogenous human airway mucus reported to date. 
\begin{figure}[h]
\centering
\includegraphics[scale=0.9]{figures/fig1_AirwayDefense.eps}
\caption{\textbf{Osmotic pressure of mucus is an important measure of human
lung health.} (a) The mucus clearance system of human airways contains two
parts: a brushlike periciliary layer (PCL) and a gellike mucus. The PCL has
an inherent osmotic pressure due to the repulsion between the mucin
molecules attached to cilia and airway surface. For healthy case the mucus
osmotic pressure is relatively low ($\sim 300$ Pa). The brushlike
periciliary layer is well hydrated and effective mucus clearance is
maintained. (b) For patients with lung diseases, including COPD and CF, the
mucus osmotic pressure (concentration) is much higher ($\sim 1000-3000$ Pa)
and the PCL is dehydrated, resulting in impaired mucus clearance of human
airways.}
\label{fig:fig1_AirwayDefense}
\end{figure}

We herein present the results of the measurements of osmotic pressure of
endogenous human airway mucus using a custom-designed membrane osmometer. We
systematically measure the osmotic pressure of mucus with concentration
ranging from $0.002$ g/ml to $0.14$ g/ml, which covers the condition for
health ($\sim 0.02$ g/ml), COPD ($\sim 0.04-0.06$ g/ml)\cite{Hogg2004}, and
CF ($>0.07$ g/ml) \cite{Tarran2004, LW1963}. We find that the osmotic
pressure \ of endogenous mucus is almost independent of pore size of the
semipermeable membrane that separates mucus and reference buffer solution,
suggesting that likely there are few free molecules contained in endogenous
mucus. Moreoever, the osmotic pressure of endogenous mucus has a linear
dependence of concentration well above the overlap concentration of mucins,
the major molecules that form the mucus framework. This linear dependence
continues to mucus concentration comparable to that of COPD ($0.06$ g/ml)
and crosses over to a stronger dependence with higher power ($2.2$) at
concentrations of CF mucus ($>0.07$ g/ml). The osmotic behavior of
endogenous mucus reveals the robustness of human airway defense: The mucus
clearance is likely to function within a reasonably wide range of mucus
concentrations, within which the mucus\ osmotic pressure has slow linear
dependence of concentration, and therefore, does not increase significantly
enough to fully collapse the periciliary layer; it only completely fails for
patients with lung diseases accompanied with extremely high mucus osmotic
pressure, such as cystic fibrosis (CF), at which the periciliary layer is
collapsed. Furthermore, the consistency between osmotic pressure of
endogenous mucus and that of clinical sputum from patients highlights that
mucus/sputum osmotic pressure could be a medical indication for the health
status of the human lung.

\begin{figure}[h]
\centering
%[height=3cm]
\includegraphics[scale=1]{figures/fig2_osmometer.eps}
\caption{\textbf{Schematic illustration of custom-designed membrane
osmometer.} Left: The osmometer has a buffer chamber (filled with PBS) and a
sample chamber separated by a semipermeable membrane with known pore size,
characterized by molecular weight cut off (MWCO). The buffer chamber is in
contact to a sensitive pressure transducer. The seal cover with a hollow
tube (diameter $d=1$ mm) is used to reduce the evaporation of the sample
while keeping the sample chamber in contact with air to maintain the
atmospheric pressure in the sample chamber. Right: 3D illustration showing
that the edge of osmotic membrane is covered by a vacuum sealant to reduce
the effective area of osmotic membrane to make sure it is fully covered by
endogenous mucus.}
\label{fig:osmometer}
\end{figure}

We obtain endogenous mucus by culturing human bronchial epithelium (HBE)
cells on an air-liquid interface\cite{Fulcher2005}, separated by a
culture-insert membrane (Transwell-Clear; Corning Costar, Cambridge, MA).
Mucins that are secreted by HBE cells accumulate on the surface of HBE
culture to form a mucus gel. The mucus gel is not perturbed except for
slowly addition of phosphate buffered saline (PBS). To measure the osmotic
pressure of the accumulated endogenous mucus, we use a custom-designed
osmometer, as illustrated in Figure \ref{fig:osmometer}, by placing the
culture-insert membrane that is carefully excised with a scalpel directly
onto semipermeable polyethersulfone membrane with $25$ mm in diameter
(Millipore Inc., Bedford, MA). In the osmometer mucus is separated from
phosphate buffered saline (PBS) by the semipermeable membrane with known
pore size denoted by molecular weight cut off (MWCO), corresponding to the
lowest molecular weight globular protein (in daltons) that is $90\%$
retained by the membrane. The PBS is sealed in a chamber that is in contact
with a sensitive pressure transducer (Omega Engineering, Stamford, CT),
affixed to the bottom of the chamber. The osmotic pressure describes the
tendency of solvent molecules to move through a semi-permeable membrane and
into a solution containing a solute to which is impermeable. In our
experiments it corresponds to the negative pressure applied to the buffer
chamber that prevents the flow of water to mucus across the semipermeable
membrane. We calibrate the osmometer with commercial osmotic pressure
standards (Wescor Inc., Logan, UT) before measurements.

We first investigate the dependence of mucus osmotic pressure on the MWCO of
a semipermeable membrane. The hypothesis is that the osmotic pressure of
mucus decreases with the increase of membrane pore size (MWCO). This is
because mucus is believed to be a polydisperse gel that contains molecules
with mass ranging from tens of kilo Daltons to millions of Daltons; as the
pore size of the semipermeable membrane increases, more molecules in the
mucus may pass through the membrane. These membrane-passing molecules do not
contribute to the osmotic pressure of mucus. 
\begin{figure}[h]
\centering
%[height=3cm]
\includegraphics[scale=1]{figures/fig3_MWCO.eps}
\caption{\textbf{Osmotic pressure of endogenous mucus depends weakly on
molecular weight cut off (MWCO) of semipermeable membrane}. The pressure of
endogenous mucus is almost independent of MWCO of semipermeable membrane for
relatively low concentration ($0.03$ g/ml) (diamonds); as the concentration
becomes larger (circles, squares, and triangles), the osmotic pressure
slightly decreases, less than $30\%$, even the MWCO of membrane increases by 
$50$ times from $10$ kDa to $500$ kDa. Data are shown as mean$\pm $SD with
the number of samples from parrallel cell cultures ($n=3$ to $5$). }
\label{fig:MWCO}
\end{figure}

Our measurement for osmotic pressure of endogenous mucus, however, argues
against this hypothesis. To perform osmotic measurements we use osmotic
membranes with molecular weight cutoff (MWCO) of $10$ kDa (pore diameter
about $2.8$ nm extrapolated from the data in ref. \citenum{Singh1998}), $100$
kDa (pore diameter $\sim 11$ nm measured by solute transport methods\cite%
{Singh1998}), and $500$ kDa (pore diameter $\sim 28$ nm extrapolated from
the data in ref. \citenum{Singh1998}). The osmotic pressure of endogenous
mucus has a very weak dependence on the MWCO of semipermeable osmotic
membrane that covers a wide range from $10$ kDa to $500$ kDa. In fact, the
pressure at relatively lower mucus concentration of $0.03$ g/ml is almost
independent of the pore size of semipermeable membrane, as shown by the
diamonds in Figure \ref{fig:MWCO}. Moreover, only up to a high mucus
concentration $\sim 0.10$ g/ml, which is larger than the typical
concentration of COPD mucus, a slightly decrease in osmotic pressure is
observed, as shown by the squares in Figure \ref{fig:MWCO}. Even at very
high concentration of $0.14$ g/ml, the osmotic pressure of endogenous mucus
decreases by only about $30\%$ as the MWCO of semipermeable membrane
increases by $50$ times from $10$ kDa to $500$ kDa.

This is surprising as only up to $30\%$ (wt/wt) of the solid contents in
mucus is contributed by mucins, the rest are mostly non-mucin proteins,
whose molecular weight is typically below $500$ kDa\cite{Rose1992,
Thornton2008, Fahy2010}. Therefore, they are expected to be able to freely
pass through the $500$ kDa semipermeable membrane and thus do not contribute
to the osmotic pressure of mucus when measured using $500$ kDa membrane.
Consequently, the existence of free small molecules in mucus is expected to
result in the phenomenon that as the MWCO of semipermeable membrane
increases, the osmotic pressure of the mucus decreases tremendously. To
estimate the effect of membrane-passing molecules on mucus osmotic pressure,
we consider mucus as a polydisperse gel with concentration well above the
overlap value. At such condition, the osmotic pressure $\pi $ of a polymer
gel is independent of polymer molecular weight and is proportional to the
concentration $c$ with a power law due to repulsion between overlapping
polymers: $\pi \propto c^{2.3}$.\cite{Rubinstein2003} Therefore, reducing
the amount of moelecules that effectively contribute to the osmotic pressure
by $70\%$ leads to decrease of pressure by more than $90\%$. However, this
prediction is found to be contrary to our results for osmotic pressure of
endogenous mucus measured by semipermeable membrane with different values of
MWCO. We therefore conclude that most small molecules are bonded to the
network framework of endogenous mucus, which allows endogenous mucus to form
a concrete gel.

Considering that the osmotic pressure of endogenous mucus has a weak
dependence on the pore size of semipermeable membrane, we use a membrane
with MWCO of $100$ kDa to systematically measure mucus osmotic pressure at
different concentrations. To obtain endogenous mucus with different
concentrations, parallel cultures were exposed to various amounts of
exogenous fluid ($5-50$ $\mu $l of PBS) approximately $1$ hour before the
osmotic pressure measurements. The obtained mucus pressure data is termed
\textquotedblleft on-cell\textquotedblright , as shown by empty circles in
Figure \ref{fig:salt}. To measure the mucus concentration under each
experimental condition, dry-to-wet ratio experiments \cite{Tarran2004} are
performed in parallel cultures. The obtained concentration defined as the
ratio of total mass of solids in mucus to the mass of mucus, including the
salt contribution ($\sim 1\%$), is conventionally called $\%$ solids. By
subtracting $1\%$ from this value one can convert $\%$ solids to the
concentration in terms of g/ml, corresponding to the mass of solids
excluding salts per unit volume of mucus, as the density of mucus is $\sim 1$
g/ml. For instance, $2\%$ solids is equivalent to $0.01$ g/ml.

The \textquotedblleft on-cell\textquotedblright\ measurement, however, only
allows proper handling of mucus with concentration down to $0.02$ g/ml. As
the concentration of mucus becomes lower, the mucus becomes more fluidic and
falls over when inverting the excised culture insert membrane to place it on
top of osmoitc membrane (Figure \ref{fig:osmometer}). Therefore, to
investigate the mucus osmotic pressure at lower concentration, we collect
mucus with desired concentrations from parallel cell cultures and use $\sim
200$ $\mu $l for each measurement. The obtained mucus pressure is termed
\textquotedblleft off-cell\textquotedblright , as presented by empty squares
in Figure \ref{fig:salt}. We perform the osmotic measurement for
\textquotedblleft off-cell\textquotedblright\ mucus immediately after the
its collection to avoid possible degradation, as evidenced by comparable
values to that obtained using \textquotedblleft on-cell\textquotedblright\
mucus of similar concentration (squares and circles of similar
concentrations in Figure \ref{fig:salt}). 
\begin{figure}[h!]
\centering
%[height=3cm]
\includegraphics[scale=1]{figures/fig4.eps}
\caption{\textbf{Dependence of osmotic pressure for endogenous mucus and
effects of salts.} On-cell: mucus that accumulates on the cell cultures
without any perturbation except by adding PBS for dilution; both
concentration and pressure for data points (circles) are shown as mean$\pm $%
SD for mucus samples from parallel cell cultures ($n=3$ to $5$). Off-cell:
mucus that are gently collected from parallel cell cultures using pipette
and then diluted into different concentrations; concentration of data points
(squares) is mean$\pm $SD for samples with $n=3$ to $5$. The measurement of
osmotic pressure for off-cell mucus was performed immediately after
collection to avoid possible degradation of mucus. The osmotic pressure
indicates a linear dependence ($\protect\pi \sim c^{0.98\pm 0.08}$, red
dash-dotted line) of concentration up to $\sim 0.06$ g/ml and then transits
to higher power ($\protect\pi \sim c^{2.21\pm 0.17}$, red dashed line)
dependence at concentration higher than $\sim 0.08$ g/ml). The crossover
concentration between the linear and high power law ($2.3$) regimes is about 
$0.075$ g/ml. Green line represents best fit to both on-cell and off-cell
measurements of mucus osmotic pressure using phenomenological equation 
\protect\ref{eq:pi_mucus-conc}. Colored symbols: Effects of salts on osmotic
pressure of endogenous mucus. The fact that $7\%$ hypertonic saline (HS)
(triangles) has negligible effect on the osmotic pressure of endogenous
supports the current treatment to cystic fibrosis patients using HS: HS does
not alter the properties of mucus but only draws water to mucus, makes the
mucus less concentrated and thus lowers its osmotic pressure, leads to
hydration of PCL, and therefore helps recovery of mucus clearance. However,
120 mM CaCl$_{2}$ (stars) leads to decrease of mucus osmotic pressure, but
slightly by $\sim 20\%$, possibly due to stronger screening of electrostatic
repulsion between charged groups on mucins by divalent Ca$^{2+} $ ions. }
\label{fig:salt}
\end{figure}

We measured the osmotic pressure of endogenous mucus with concentration
ranging from $0.002$ g/ml to $0.14$ g/ml that covers the values spanning
from health condition ($0.02-0.04$ g/ml), COPD ($0.04-0.06$ g/ml)\cite%
{Hogg2004}, and CF ($>0.07$ g/ml) \cite{Tarran2004, LW1963}. We identify two
regimes for the concentration dependent mucus osmotic pressure. As shown in
Figure \ref{fig:salt}, within the low concentration regime (from $\sim 0.002$
g/ml to $\sim 0.06$ g/ml), the osmotic pressure of mucus has a linear
dependence on concentration $\pi \sim c^{\alpha }$, in which $\alpha
=0.98\pm 0.08$ as determined by the best fit to data up to $0.06$ g/ml at
logarithmic scales for both axes, as shown by dash-dotted line in Figure \ref%
{fig:salt}. In the high concentration regime from $\sim 0.08$ g/ml to $\sim
0.14$ g/ml, the osmotic pressure increases as a higher power of
concentration $\pi \sim c^{\alpha }$, where $\alpha =2.21\pm 0.17$ is
obtained by the best fit to data above $0.08$ g/ml at logarithmic scales for
both axes, as shown by the dashed line in Figure \ref{fig:salt}. The
crossover mucus concentration for these two regimes is about $c^{\ast
}\simeq 0.075$ g/ml.

To understand the behavior of osmotic pressure for endogenous mucus, we
first back to the well-understood osmotic pressure of regular polymer
solutions. Typically osmotic pressure of polymer solutions in a good solvent
can be described by a crossover phenomenological equation \cite%
{Rubinstein2003}

\begin{equation}
\pi \simeq \frac{N_{Av}k_{B}T}{M}c[1+(c/c^{\ast })^{\alpha -1}]
\label{eq:pi}
\end{equation}%
where $N_{Av}$ is the Avogadro number, $k_{B}$ is the Boltzmann constant, $T$
is the absolute temperature, $M$ corresponds to the number average molar
mass of polymer, and $c^{\ast }$ is the polymer overlap concentration. The
linear component in eq. \ref{eq:pi} represents the van't Hoff law for
osmotic pressure of dilute solutions, in which the contribution of molecules
to osmotic pressure is about $k_{B}T$ per molecule. The higher power terms
accounts for the repulsion between overlapping polymers. Typically the value
of exponent $\alpha $ is about $2.25$ for linear or branched polymers in
good solvent \cite{Rubinstein2003}.

It seems that the two-regime behavior of mucus osmotic pressure is
consistent with the two-regime prediction from regular polymer solutions.
However, the crossover concentration, $0.08$ g/ml, for endogenous mucus is
more than one order of magnitude larger than the estimated value: $c^{\ast
}\simeq M/\left( 4\pi R^{3}/3\right) \simeq \lbrack 10\times 10^{6}$ g/molar$%
]/[4\pi (100$ nm$)^{3}/3]\simeq 0.004$ g/ml, in which $M\simeq 10\times
10^{6}$ g/molar is the typical molecular weight of mucin molecules, $R\simeq
100$ nm corresponds to the size of mucin moecules.\cite{Bansil1995,
Sheehan2000,Kesimer2010} This inconsistency suggests that the besides the
repulsion between overlapping molecules, there should be some other facts
that contribute to the osmotic pressure of mucus.

Indeed, mucin molecules are charged biological polymers; moreoever, there
are interacting groups between mucin molecules. Therefore, the osmotic
pressure of mucus $\pi $ is contributed from three parts: ionic
contribution, $\pi _{ion}$, polymeric contribution due to repulsion between
molecules, $\pi _{poly}$, and contribution from interaction between
molecules, $\pi _{int}$:

\begin{equation}
\pi =\pi _{ion}+\pi _{poly}+\pi _{int}  \label{eq:pi_mucus-1}
\end{equation}%
The ionic contribution is $\pi _{ion}/k_{B}T\simeq zc+z^{2}c^{2}/\left(
4c_{s}\right) $, in which $c$ is the number concentration of charged
molecules, $z$ is the uncondensed charge groups per charged molecule, and $%
c_{s}$ is the number of concenration of salts.\cite{Dobrynin2005} The
polymeric contribution is about $k_{B}T$ per correlation volume (eq. \ref%
{eq:pi}): $\pi _{poly}/k_{B}T\simeq 1/\xi ^{3}$.\cite{Rubinstein2003} Here
the correlation length $\xi $ is proportional to concentration $c$ by a
power law in good solvent: $\xi \approx R(c/c^{\ast })^{-0.76}\approx
R\left( c_{n}/c_{n}^{\ast }\right) ^{-0.76}$, in which $R$ is the size
molecules in mucus. The interaction part tends to reduce the osmotic
pressure of mucus by holding the molecules together; its contribution is
proportional to the number density of interacting groups (pair of
interacting sites). The probability for an interacting site to form a pair
with another interacting site is proportional to $c^{2}$. Therefore, the
contribution to mucus osmotic pressure from the interaction between
molecules in the mucus is: $\pi _{ion}/k_{B}T\simeq -\gamma c^{2}$, in which 
$\gamma $ is a parameter that corresponds to the volume occupied by a
molecule. Therefore, the osmotic pressure of mucus (eq. \ref{eq:pi_mucus-1})
can rewritten as

\begin{equation}
\pi /k_{B}T\simeq zc_{n}+\frac{z^{2}c_{n}^{2}}{4c_{n}^{s}}+\frac{1}{R^{3}}%
\left( \frac{c_{n}}{c_{n}^{\ast }}\right) ^{2.3}-\gamma c_{n}^{2}
\label{eq:pi-number}
\end{equation}%
Note that in eq. \ref{eq:pi-number} $c_{n}$ has the unit of the number of
molecules per volume. In our experiments the concentration $c$ is measured
in terms of g/ml. Denoting the average mass of molecules $M$ in the mucus by
million Dalton ($10^{6}$ g/molar), the relation between concentration $c_{n}$
and concentration $c$ is: $c_{n}=\left( cN_{Av}/M\right) \unit{m}^{-3}$.
Considering the small difference between power $2.3$ and $2.0$, we
approximate the osmotic pressure of mucus (eq. \ref{eq:pi-number}) by two
parts%
\begin{eqnarray}
\pi &\simeq &\frac{k_{B}TN_{Av}}{m^{3}}\left[ \frac{z}{M}c+\lambda
_{2}\left( \frac{c}{c^{\ast }}\right) ^{2.3}\right]  \nonumber \\
&\simeq &2500Pa\left[ \lambda _{1}c+\lambda _{2}\left( c/c^{\ast }\right)
^{2.3}\right]  \label{eq:pi_mucus-conc}
\end{eqnarray}%
in which $c$ has the unit of g/ml, $\lambda _{1}=z/M$ stands for the number
of charged groups per million Dalton, $\lambda _{2}$ represents the
contribution from the polymeric repulsion, attraction between molecules, and
counterions. The value of $\lambda _{1}=5.1$ is determined by weighted fit
to the data with concentration below $0.06$ g/ml. Using the predetermined
value of $\lambda _{1}=5.1$, $\lambda _{2}=0.22$ and $c^{\ast }=0.08$ g/ml
are determined by the best fit to all data points, as shown by the solid
line in Figure \ref{fig:salt}. Note that $c^{\ast }$ is the effective
crossover concentration that takes into account the contribution from
interacting sites and counterions; it is shifted to higher value in
comparison to that estimated from polymeric contribution. This
phenomenological equation captures the main feature of the mucus osmotic
pressure, as presented by the solid line in Figure \ref{fig:salt}. 
\begin{figure}[h!]
\centering
%[height=3cm]
\includegraphics[scale=1]{figures/fig5_clinic.eps}
\caption{\textbf{Osmotic pressure of sputum from patients.} Symbols: osmotic
pressure measured by using semipermeable membranes with MWCO of $10$ kDa
(empty circles) and $100$ kDa (filled circles). Lines: osmotic pressure of
endogenous mucus that is adapted from Figure \protect\ref{fig:salt}. The
osmotic of clinical sputum is in qualitatively agreement with that of
endogenous mucus, which highlights that osmotic pressure of mucus/sputum
could be used as a measure for the health status of human lungs.}
\label{fig:clinic}
\end{figure}

We also exploit the effect of the salts on mucus osmotic pressure. We adjust
the salt concentration of mucus using hypertonic saline; meanwhile, we
replace the PBS in the buffer chamber with solution of the same
concentration of salts as mucus. At the physiological condition with salt
concentration of $150$ mM the Debye screening length is about $r_{D}\simeq
0.9\unit{nm}\propto c_{s}^{-1/2}$\cite{Dobrynin2005}. Increasing the salt
concentration to $7\%$ results in further decrease of Debye screening length
down to $0.3$ nm. However, notice that the cross dimension of mucin
molecules is about $5$ nm\cite{Jentoft1990}, which is more than $5$ times
larger than the Debye screening length of at physiological condition.
Therefore, increasing the extent of salt screening do not affect the osmotic
pressure contributed from counter ions. Consistent with our expectation, the
addition of monovalent salt ions (NaCl) almost does not affect mucus osmotic
pressure. This is not surprising as the charged groups are highly screened
by salt ions. This finding favorably reasons the physical basis for using
hypertonic saline as one of the effective and major treatments to patients
with lung diseases, especially CF. Adding more salts might not alter the
structure of mucus, but draws water into mucus, which results in dilution of
mucus and therefore lower osmotic pressure, and consequently, improves the
clearance of mucus. However, introducing divalent ions by adjusting mucus
with $120$ mM CaCl$_{2}$ leads to decrease in osmotic pressure of $\sim 20\%$%
, as shown by the stars in Figure \ref{fig:salt}. This might be because
divalent ions may mediate the repulsion between molecules in solution in a
stronger way comparing to the monovalent ions. This finding is consistent
with the fact that mucin molecules are densely packed within granules before
they are released, where they are surrounded by extensive calcium ions\cite%
{Kesimer2010}. A complete understanding of mucus osmotic pressure and the
effects of salts on their osmotic pressure remains a puzzle requiring
further inquiry of fundamental polymer physics properties of mucin molecules.

Finally we compare the osmotic pressure of endogenous mucus with that of
sputum samples donated by patients. We collect sputum from patients with
different diseases, measure the sputum concentration, and quantify the
osmotic pressure. We measure the pressure using semipermeable membranes of
two different molecular weight cut-off, $10$ kDa and $100$ kDa. We find that
osmotic pressure of clinical mucus for $100$ kDa is systematically lower
than that measure using $10$ kDa, as shown in Figure \ref{fig:clinic}, which
is different from that for endogenous mucus, as shown in Figure \ref%
{fig:MWCO}. This is because clinical samples are not as \textquotedblleft
clean\textquotedblright\ as endogenous mucus. Clinical sputum samples
contain lots of cell debris, blood residues, and other small molecules that
may pass through the osmotic membrane as its MWCO increases. Therefore, the
osmotic pressure of sputum decreases with increase of MWCO of semipermeable
membrane. Nevertheless, the interesting behavior is that the pressure of
clinical sputum is in qualitatively agreement with that of endogenous mucus,
as shown by the solid line in Figure \ref{fig:clinic}. The consistency in
trend of osmotic pressure for endogenous mucus from cell cultures and
clinical sputum suggests osmotic pressure of mucus/sputum could be an
important medical indication of the health status of human lung.

Our work highlights the robustness of human airway defense. It can function
within a wide range of mucus concentrations because of relatively slow
increase of mucus osmotic pressure at relatively low concentrations.
Importantly, osmotic pressure of mucus/sputum could be used an medical
indication to monitor the health status of lung for patients. Furthermore,
the striking difference in osmotic pressure of endogenous mucus from regular
polymer solutions highlights the unique properties that mucus may posses. It
also brings attention to use endogenous, instead of reconstituted mucus,
when studying its unique properties, such as vicoelastic properties that is
critical to maintain effective interaction between mucus and cilia\cite%
{Litt1970} and serving as a protective barrier by trapping external objects%
\cite{Knowles2002}.

\textbf{Acknowledgement.} This article is dedicated to Prof. John K.
Sheehan, for years of support, fruitful discussions and his insights into
mucus. We acknowledge financial support from the National Science Foundation
under grants CHE-0911588, DMR-0907515, DMR-1121107, DMR-1122483, and
CBET-0609087, the National Institutes of Health under grants R01HL077546 and
P50HL107168, and Cystic Fibrosis Foundation under grant RUBIN09XX0.

%The conclusions section should come at the end of article. For the reference
%section, the style file rsc.bst can be used to generate the correct
%reference style.\footnote[4]{%
%Footnotes should appear here. These might include comments relevant to but
%not central to the matter under discussion, limited experimental and
%spectral data, and crystallographic data.}
%For footnotes in the main text of the article please number the footnotes to avoid duplicate symbols. e.g.  \footnote[num]{your text} the corresponding author \ast counts as footnote 1, ESI as footnote 2, e.g. if there is no ESI, please start at [num]=[2], if ESI is cited in the title please start at [num]=[3] etc. Please also cite the ESI within the main body of the text using \dag.

%The \balance command can be used to balance the columns on the final page if desired. It should be placed anywhere within the first column of the last page.

%\balance

%If notes are included in your references you can change the title from 'References' to 'Notes and references' using the following command:
%\renewcommand\refname{Notes and references}

{\footnotesize {\ 
\bibliographystyle{rsc}
\bibliography{mucusOsmotic}
%the RSC's .bst file
} }

%\footnotesize{
%\bibliography{rsc} %your .bib file
%\bibliographystyle{rsc} %the RSC's .bst file
%}

\end{document}
